\chapter*{Introduction}
\label{chap:Introduction}
\addcontentsline{toc}{chapter}{Introduction}

	\section*{Objet du rapport}
L'objet de ce document est la modélisation d'une application d'informatisation d'un entrepôt de stockage avec UML et l'aide d'un AGL.
	Nous allons tout d'abord spécifier les besoins en détailles cette phase a pour but de décrire précisément :
	\begin{enumerate}[label=\textbullet]
	\item L'ensemble des fonctionnalités de l'application.
	\item Les objets manipulés, leurs buts et leurs principes de fonctionnement.
	\end{enumerate}
	
	Dans la 2éme partie du rapport nous allons nous focaliser sur l'analyse des besoins, cette phase nous permettra de bien comprendre le contexte et de déterminer les besoins et les contraintes.
	
	La 3éme partie consiste en "La conception" du system. Nous établissons les diagrammes de communication qui sont utilisés pour décrire entre les objets de notre system, enfin le diagramme de classes de conception.
	
	La 4éme partie concerne la génération du code java à partir du diagramme de classe de conception. Nous présentons l'AGL qui nous permettra de générer du code de façon automatique.
	
	\section*{Domaine du system}
Ce rapport est applicable pendant la phase de développement du system « Informatisation d'un entrepôt de stockage.
La modélisation de ce system sera conforme aux éléments présents dans ce rapport.


