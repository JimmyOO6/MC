\pagestyle{fancy}
\chead{}
\rhead{}
\lfoot{}
\cfoot{\thepage}
\rfoot{}

\section{Spécification des besoins}
	\subsection{Identification des acteurs}
	On entend par acteur, un humain, une machine ou un système qui ne fait pas partie de la solution à réaliser mais qui participe à son fonctionnement général par une interaction. Dans notre cas, nous aurons typiquement des acteurs humains qui sont des employés et du superviseur (Voir tableau 2.1).
	
\begin{table}[!ht]

		\begin{tabular}{|c|l|}
		\hline
		\textbf{Acteur} & \textbf{Rôle} \\
		\hline
		Employé & L'employé aura accès aux fonctionnalités du logiciel après authentification\\
		\hline
		Superviseur & En plus d'avoir aura accès aux fonctionnalités du logiciel il pourra gérer\\
		& la bonne application des consignes\\
		\hline
		Administrateur & Gérer les utilisateurs\\
		\hline
		Système de &  Recevoir des informations à l'issue de la réalisation du cas d'utilisation\\gestion &des autre acteurs\\
		\hline
		\end{tabular}
		\caption{Les acteurs du système}
	\end{table}
	
	\subsection{Modélisation du contexte}
	La figure 2.1 représente les interactions entre le système et les acteurs qui y sont impliqués :
	
	\begin{figure}[!h]
		\center
			\includegraphics[scale=1.2]{images/DC}
			\caption{Diagramme de contexte du système à réaliser.}
	\end{figure}
	
	
	\clearpage
	\subsection{Identifications des cas d’utilisation}
	Le Cas d'utilisation est une description des interactions qui permettront à l'acteur d'atteindre son objectif en utilisant le système \cite{UML}. Le tableau 2.2 résume les cas d’utilisations du système à réaliser :\\
	\begin{table}[!h]
	
  	\begin{center}
	\begin{tabular}{|c|c|c|c|}
	\hline
	\textbf{N}& \multicolumn{2}{|c|}{\textbf{Cas d’utilisation}} & \textbf{Acteur}\\
	\hline
	1 & \multicolumn{2}{|c|} {S'authentifier} & \\
	\cline{1-3}
	2 & \multicolumn{2}{|c|} {Saisie des caractéristiques d'articles lors du chargement} & Employé/Superviseur\\
	\cline{1-3}
	3 & \multicolumn{2}{|c|} {Saisie des caractéristiques d'articles lors du déchargement} & \\
	\hline
	4 & \multicolumn{2}{|c|} {Gérer les employés} & Superviseur\\
	\hline
	\end{tabular}
	\caption{Cas d’utilisation du système à réaliser}
\end{center}
\end{table}
\subsection{Diagramme de cas d'utilisation}
	\begin{figure}[!h]
		\center
			\includegraphics[scale=0.65]{images/DCD}
			\caption{Diagramme de contexte du système à réaliser.}
	\end{figure}
	\clearpage
	
	\section{Analyse des besoins}
	
\subsection{Description des cas d'utilisation}
	Les diagrammes réalisés jusqu'à maintenant (diagramme de contexte, diagramme de cas d'utilisation) nous ont permis de découvrir petit à petit les fonctionnalités que l'on devrait avoir dans le futur logiciel.
	
	Nous allons désormais parler de l'interaction entre les acteurs et le système : il s'agit de décrire la chronologie des actions qui devront être réalisées par les acteurs et par le système lui-même. Cette description va nous permettre :
	
\begin{enumerate}[label=\textbullet]
	\item clarifier le déroulement de la fonctionnalité.
	\item décrire la chronologie des actions qui devront être réalisées.
	\item d'identifier les parties redondantes pour en déduire des cas d'utilisation plus précises qui seront utilisées par inclusion, extension ou généralisation/spécialisation. Et oui, dans ce cas nous réaliserons des itérations sur les diagrammes de cas d'utilisation.
	\item d'indiquer d'éventuelles contraintes déjà connues et dont les développeurs vont devoir tenir compte lors de la réalisation du logiciel. Ces contraintes peuvent être de nature diverse.
	\end{enumerate}
	
	\subsubsection*{Cas d'utilisation <<S'authentifier>> :}
	\textbf{Cas d'utilisation} :  <S'authentifier>
	
	\textbf{Acteurs} :  <Employé,Superviseur>
	
	\textbf{Objectif} : -Il permet à l'acteur de s'authentifier.
	
	\textbf{[Pré-condition :]}  -Les identifiants de l'acteur doivent exister dans la base de donnée.
	
	\textbf{[Post-condition :]}  -Acteur Identifié	
	
	\textbf{Scénario nominal :}  1. L'acteur ouvre le logiciel.
	  
    2. Le système affiche la fenêtre d'authentification
      
    3. L'acteur saisit les identifiants
    
    4. Le système vérifie l'existence des données
     
    5. Le système identifie l'acteur.  
    \clearpage
    \textbf{Scénario alternatif :} \textit{A. Erreur d'authentification :} identifiants non valides. 
    
Cet enchaînement démarre au point 4.
   
5. Le système affiche un message d'erreur. 

Le scénario reprend au point 2.

\textbf{[Contraintes non fonctionnelles :] } <Confidentialité> 
    
    \subsubsection*{Cas d'utilisation <<Saisie des caractéristiques d'articles lors du déchargement>> :}
	\textbf{Cas d'utilisation} :  <Saisie des caractéristiques d'articles lors du déchargement>
	
	\textbf{Acteurs} :  <Employé,Superviseur>
	
	\textbf{Objectif} : - Il permet à l'acteur de 	de générer une liste où figure un emplacement 	pour 
	
	chaque article saisi.
	
	\textbf{[Pré-condition :]}  - L'article doit 	exister dans la base de donnée.
	
	\textbf{[Post-condition :]}  -Génère une 			liste où figure un emplacement pour chaque 
	
	article saisi.
	
	\textbf{Scénario nominal :} 1. L'acteur 			saisit les caractéristique de l'article.
    
    2. Le système vérifie l'existence de 				l'article.
    
     3. Le système recherche un emplacement dans 		un stock.
     
    4. Le système affiche la liste où figure un emplacement pour chaque article.  
    
    \textbf{Scénario alternatif :} \textit{A. Erreur article non-existant :} caractéristique invalide. 
    
Cet enchaînement démarre au point 2.
   
5. Le système affiche un message d'erreur. 

Le scénario reprend au point 1.

\textbf{[Contraintes non fonctionnelles :] } <Temps de réponse>

